\begin{tikzpicture}[scale=0.5]
    \draw[fill=gray!20] (0,0) rectangle (4,4);
    \draw[step=1cm, gray!50, thin] (0,0) grid (4,4);
    
    % Шахматная конфигурация
    \foreach \x in {0.5,1.5,2.5,3.5}
        \foreach \y in {0.5,1.5,2.5,3.5} {
            \pgfmathtruncatemacro{\parity}{mod(\x+\y,2)}
            \ifnum\parity=0
                \draw[->, thick, blue] (\x,\y) -- (\x,\y+0.4);
            \else
                \draw[->, thick, red] (\x,\y) -- (\x,\y-0.4);
            \fi
        }
    
    % Название сверху
    \node[above, font=\footnotesize, align=center] at (2,4.2) {Антиферромагнитная\\конфигурация};
    
    % Подпись снизу
    \node[below, font=\scriptsize, align=center] at (2,-0.2) {Чередующиеся\\направления спинов};
    
    % Энергия - опущена ещё ниже
    \node[draw, fill=yellow!20, rounded corners=0.2cm, font=\scriptsize, align=center] at (2,-3.0) {
        $E = -\sum_{<ij>} J S_i S_j$, $J < 0$, $S_i S_j = -1$
    };
\end{tikzpicture}