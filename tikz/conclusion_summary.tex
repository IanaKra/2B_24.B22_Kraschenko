\begin{tikzpicture}[scale=0.9]
    % Основной круг с секторами (увеличенный радиус в 1.3 раза)
    \foreach \angle/\color in {0/red!30, 90/green!30, 180/yellow!30, 270/cyan!30}
        \draw[fill=\color] (0,0) -- (\angle:3.9) arc (\angle:\angle+90:3.9) -- cycle;
    
    % Центральный узел
    \node[draw, circle, fill=white, minimum size=1.5cm] at (0,0) {АНС};
    
    % Подписи секторов - на углах 45°, 135°, 225°, 315° (восстановлен шрифт)
    \node[align=center, font=\small\bfseries, text width=1.5cm] at (45:2.3) {Емкость\\памяти};
    \node[align=center, font=\small\bfseries, text width=1.5cm] at (147:2.8) {Точность\\восстановления};
    \node[align=center, font=\small\bfseries, text width=1.5cm] at (217:2.6) {Устойчивость\\к шуму};
    \node[align=center, font=\small\bfseries, text width=1.5cm] at (315:2.2) {Скорость\\сходимости};
    
    % Внешние блоки - расположены слева и справа (восстановлен шрифт)
    % Левые блоки
    \node[draw, rectangle, fill=green!20, rounded corners=0.3cm,
          minimum width=2.8cm, minimum height=1.8cm, font=\small, align=center] 
          (achievements) at (-7, 2) {
            Достижения \\
            Ассоциативная память \\
            Самокоррекция
    };
    
    \node[draw, rectangle, fill=yellow!20, rounded corners=0.3cm,
          minimum width=2.8cm, minimum height=1.8cm, font=\small, align=center] 
          (perspectives) at (-7, -2) {
            Перспективы \\
            Большие сети \\
            Биологические данные
    };
    
    % Правые блоки
    \node[draw, rectangle, fill=red!20, rounded corners=0.3cm,
          minimum width=2.8cm, minimum height=1.8cm, font=\small, align=center] 
          (limits) at (7, 2) {
            Ограничения \\
            Ложные аттракторы \\
            Малая емкость
    };
    
    \node[draw, rectangle, fill=cyan!20, rounded corners=0.3cm,
          minimum width=2.8cm, minimum height=1.8cm, font=\small, align=center] 
          (applications) at (7, -2) {
            Применения \\
            Нейроморфные системы \\
            Распознавание образов
    };
    
    % Горизонтальные стрелки от круга к внешним блокам (пересчитаны для нового радиуса)
    % Левые блоки - стрелки идут влево из точек на круге
    \draw[->, thick] (-3.35, 2) -- (achievements.east);
    \draw[->, thick] (-3.35, -2) -- (perspectives.east);
    
    % Правые блоки - стрелки идут вправо из точек на круге
    \draw[->, thick] (3.35, 2) -- (limits.west);
    \draw[->, thick] (3.35, -2) -- (applications.west);
    
    % Точки на круге для привязки стрелок (опционально)
    \fill[black] (-3.35, 2) circle (1.5pt);
    \fill[black] (-3.35, -2) circle (1.5pt);
    \fill[black] (3.35, 2) circle (1.5pt);
    \fill[black] (3.35, -2) circle (1.5pt);
\end{tikzpicture}