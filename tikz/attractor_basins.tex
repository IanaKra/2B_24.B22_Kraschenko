\begin{tikzpicture}
    \begin{axis}[
        width=0.7\textwidth,
        height=8cm,
        xlabel={Координата 1 (t-SNE 1)},
        ylabel={Координата 2 (t-SNE 2)},
        title={Бассейны аттракторов в фазовом пространстве},
        grid=major,
        xmin=-3, xmax=3,
        ymin=-3, ymax=3,
        legend style={
            at={(1.05,0.5)},
            anchor=west,
            legend columns=1,
            cells={anchor=west},
            draw=none,
            row sep=3pt,
        },
        enlargelimits=0.1
    ]
        
        % Аттрактор A - КРАСНЫЙ
        \addplot[
            only marks,
            mark=*,
            mark size=2pt,
            color=red,
        ] coordinates {
            (-2, -2) (-1.8, -1.9) (-2.1, -1.7) 
            (-1.9, -2.2) (-2.2, -2.1) (-1.7, -1.8)
        };
        \addlegendentry{Аттрактор A}
        
        \addplot[
            red,
            thick,
            domain=0:2*pi,
            samples=50,
            smooth
        ] ({-2+0.8*cos(deg(x))}, {-2+0.8*sin(deg(x))});
        \addlegendentry{Бассейн A}
        
        \node[red, align=center, font=\bfseries] at (-2, -2.9) {36.3\%};
        
        % Аттрактор B - СИНИЙ
        \addplot[
            only marks,
            mark=*,
            mark size=2pt,
            color=blue,
        ] coordinates {
            (2, 2) (1.8, 1.9) (2.1, 1.7)
            (1.9, 2.2) (2.2, 2.1) (1.7, 1.8)
        };
        \addlegendentry{Аттрактор B}
        
        \addplot[
            blue,
            thick,
            domain=0:2*pi,
            samples=50,
            smooth
        ] ({2+0.8*cos(deg(x))}, {2+0.8*sin(deg(x))});
        \addlegendentry{Бассейн B}
        
        \node[blue, align=center, font=\bfseries] at (2, 1.1) {34.6\%};
        
        % Ложный аттрактор 1 - ЗЕЛЁНЫЙ
        \addplot[
            only marks,
            mark=*,
            mark size=2pt,
            color=green!70!black,
        ] coordinates {
            (0, 0.8) (0.2, 1.0) (-0.2, 1.2)
            (0.1, 0.9) (-0.1, 1.1)
        };
        \addlegendentry{Ложный аттр. 1}
        
        \addplot[
            green!70!black,
            thick,
            domain=0:2*pi,
            samples=50,
            smooth
        ] ({0+0.4*cos(deg(x))}, {1+0.4*sin(deg(x))});
        \addlegendentry{Бассейн ЛА1}
        
        \node[green!70!black, align=center, font=\bfseries] at (0, 1.5) {8.2\%};
        
        % Ложный аттрактор 2 - ОРАНЖЕВЫЙ
        \addplot[
            only marks,
            mark=*,
            mark size=2pt,
            color=orange,
        ] coordinates {
            (-1.5, 1) (-1.3, 1.2) (-1.7, 0.8)
            (-1.4, 1.1) (-1.6, 0.9)
        };
        \addlegendentry{Ложный аттр. 2}
        
        \addplot[
            orange,
            thick,
            domain=0:2*pi,
            samples=50,
            smooth
        ] ({-1.5+0.3*cos(deg(x))}, {1+0.3*sin(deg(x))});
        \addlegendentry{Бассейн ЛА2}
        
        \node[orange, align=center, font=\bfseries] at (-1.5, 1.4) {6.8\%};
        
        % Ложный аттрактор 3 - ФИОЛЕТОВЫЙ
        \addplot[
            only marks,
            mark=*,
            mark size=2pt,
            color=violet,
        ] coordinates {
            (1.5, -1) (1.7, -0.8) (1.3, -1.2)
            (1.6, -0.9) (1.4, -1.1)
        };
        \addlegendentry{Ложный аттр. 3}
        
        \addplot[
            violet,
            thick,
            domain=0:2*pi,
            samples=50,
            smooth
        ] ({1.5+0.3*cos(deg(x))}, {-1+0.3*sin(deg(x))});
        \addlegendentry{Бассейн ЛА3}
        
        \node[violet, align=center, font=\bfseries] at (1.5, -1.4) {5.5\%};
        
        % Несходящиеся состояния - СЕРЫЕ точки (вне всех окружностей)
        \addplot[
            only marks,
            mark=*,
            mark size=2pt,
            color=gray,
        ] coordinates {
            (0, -2.5) (2.5, -2) (-2.5, 2)
            (2.5, 2.8) (-2, 2.8) (0, 2.8)
        };
        \addlegendentry{Несходящиеся}
        
        \node[gray, align=center, font=\bfseries] at (0, 0) {8.6\%};
        
    \end{axis}
\end{tikzpicture}