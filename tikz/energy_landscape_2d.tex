\begin{tikzpicture}
    % Основной график
    \begin{axis}[
        name=mainplot,
        width=0.75\textwidth,
        height=4.5cm,
        view={0}{90},
        xlabel={Координата X},
        ylabel={Координата Y},
        grid=major,
        colormap/viridis,
        point meta min=-2,
        point meta max=2,
        at={(0,0)}
    ]
        
        % Поверхность энергетического ландшафта
        \addplot3[
            surf,
            domain=-3:3,
            domain y=-3:3,
            samples=25,
            opacity=0.7
        ] {x^4 - 2*x^2 + y^4 - 2*y^2 + 0.5*sin(deg(x*y))};
        
        % Локальные минимумы
        \addplot3[
            only marks, 
            mark=*, 
            mark size=3,
            red,
            fill=red
        ] coordinates {
            (-1.4,-1.4, {(-1.4)^4 - 2*(-1.4)^2 + (-1.4)^4 - 2*(-1.4)^2 + 0.5*sin(deg(-1.4*-1.4))})
            (1.4,1.4, {(1.4)^4 - 2*(1.4)^2 + (1.4)^4 - 2*(1.4)^2 + 0.5*sin(deg(1.4*1.4))})
            (-1.4,1.4, {(-1.4)^4 - 2*(-1.4)^2 + (1.4)^4 - 2*(1.4)^2 + 0.5*sin(deg(-1.4*1.4))})
            (1.4,-1.4, {(1.4)^4 - 2*(1.4)^2 + (-1.4)^4 - 2*(-1.4)^2 + 0.5*sin(deg(1.4*-1.4))})
        };
        
    \end{axis}
    
    % Цветовая шкала под графиком
    \begin{axis}[
        at={(mainplot.below south west)},
        anchor=north west,
        width=0.75\textwidth,
        height=0.8cm,
        scale only axis,
        hide axis,
        colormap/viridis,
        colorbar horizontal,
        colorbar style={
            width=0.75\textwidth,
            height=0.4cm,
            xticklabel pos=upper,
            ylabel={},
            yticklabel={}
        },
        point meta min=-2,
        point meta max=2,
        clip=false
    ]
        % Пустой график для цветовой шкалы
        \addplot[draw=none] coordinates {(0,0) (1,1)};
    \end{axis}
    
    % Подпись "Энергия" под цветовой шкалой (ещё ниже и дальше от подписей)
    \node[anchor=north, text=black, font=\small] at (0,-1.3) {Энергия};
\end{tikzpicture}