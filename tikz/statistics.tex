\begin{tikzpicture}
    \begin{axis}[
        width=0.9\textwidth,
        height=7cm,
        xlabel={Уровень шума (\%)},
        ylabel={Вероятность успешного восстановления (\%)},
        title={Статистика успешного восстановления образов},
        grid=major,
        legend style={
            at={(0.5,-0.25)},
            anchor=north,
            legend columns=2
        },
        xmin=0, xmax=50,
        ymin=0, ymax=105,
        xtick={0,10,20,30,40,50},
        ytick={0,20,40,60,80,100},
        scale only axis
    ]
        
        % Экспериментальные данные
        \addplot[blue, thick, mark=*, mark size=2pt] coordinates {
            (0, 100)
            (5, 99)
            (10, 97)
            (15, 92)
            (20, 85)
            (25, 75)
            (30, 62)
            (35, 45)
            (40, 28)
            (45, 15)
            (50, 8)
        };
        
        % Теоретическая кривая
        \addplot[red, thick, dashed, domain=0:50, samples=30] 
            {100/(1+exp((x-30)/5))};
        
        \legend{Экспериментальные данные, Теоретическая модель}
        
        % Области надёжности поверх сетки
        % Зеленая в диапазоне 40-60 по оси Y
        \node[draw, fill=green!20, rounded corners=0.2cm, align=center, font=\small, text width=3cm] at (8,50) {
            Высокая надёжность \\ (шум < 20\%)
        };
        
        % Желтая возвращена назад, примерно на 20-40 по оси Y
        \node[draw, fill=yellow!20, rounded corners=0.2cm, align=center, font=\small, text width=3cm] at (25,30) {
            Средняя надёжность \\ (шум 20-35\%)
        };
        
        % Красная в диапазоне 60-80 по оси Y
        \node[draw, fill=red!20, rounded corners=0.2cm, align=center, font=\small, text width=3cm] at (42,70) {
            Низкая надёжность \\ (шум > 35\%)
        };
        
    \end{axis}
\end{tikzpicture}